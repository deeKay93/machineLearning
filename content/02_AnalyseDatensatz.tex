\section{Analyse des Datensatzes}\label{sec:analyse}
Um Konzepte für lernende Systeme aufzubauen, ist es zunächst sinnvoll sich mit 
dem gegebenen Datensatz auseinander zu setzen. 
Gegeben sind 22 Attribute die jeweils eine Eigenschaft einer Wohnung darstellen. 
Die Attribute sind in verschiedenen Formen angegeben: 
\begin{itemize}
    \item Boolean-Werte: Hausmeister, Garage, Aufzug, Balkon, Terrasse, Kehrwoche, Möbliert (z.B. ja)
    \item Text-Werte: Heizung, Lage, Bad (z.B. Ölheizung)
    \item Int-Bereiche: Zimmer, Stockwerk, Miete, Nebenkosten, Alter, Entfernung, Kaution, Quadratmeter (z.B. 2-3 Zimmer)
    \item Kategorien: Kindergarten, Schule, S-Bahn, Küche (z.B. Küche(alt))
\end{itemize}

Der Datensatz zum Trainieren und Aufbauen von lernenden Systemen erhält zusätzlich 
ein Bewertungsattribut $k$. 
Auffällig ist, dass bei 400 Einträgen lediglich 68 positive Bewertungen gegeben wurden. 
Dies entspricht 17\%. 

Da für den Aufbau der Systeme der Trainingdatensatz verwendet wird, wird dieser mithilfe einer Korrelationsmatrix 
genauer analysiert. Diese zeigt die Abhängigkeiten zwischen den Eigenschaften. 

\autoref{tab:korr1} ist ein Ausschnitt der Korrelationsmatrix der eine besonders starke Korrelation aufzeigt. \\
Auffällig ist, dass zwischen Miete, Zimmerzahl und Quadratmeter ein fast vollständig linearer Zusammenhang
besteht. Auch zur Kaution besteht ein mittelstarker positiver linearer Zusammenhang.

\useunder{\uline}{\ul}{}
\begin{table}[h]
    \begin{center}
        \begin{tabular}{|l|l|l|l|l|}
            \hline
            {\ul \textbf{}}             & {\ul \textbf{Zimmerzahl}} & {\ul \textbf{Miete}} & {\ul \textbf{Nebenkosten}} & {\ul \textbf{Kaution}} \\ \hline
            {\ul \textbf{Zimmerzahl}}   & 1                         &                      &                            &                        \\ \hline
            {\ul \textbf{Miete}}        & 0,96                      & 1                    &                            &                        \\ \hline
            {\ul \textbf{Nebenkosten}}  & 0,94                      & 0,94                 & 1                          &                        \\ \hline
            {\ul \textbf{Kaution}}      & 0,58                      & 0,58                 & 0,58                       & 1                      \\ \hline
            {\ul \textbf{Quadratmeter}} & 0,98                      & 0,96                 & 0,94                       & 0,58                   \\ \hline
        \end{tabular}
        \caption{Korrelationsmatrix Zimmerzahl, Miete, Nebenkosten, Kaution}
        \label{tab:korr1}
    \end{center}
\end{table}

Auch der Ausschnitt aus \autoref{tab:korr2} zeigt einen linearen Zusammenhang zwischen Eigenschaften. Die Korrelationskoeffizienten 
zeigen hier einen mittelstarken linearen Zusammenhang zwischen Stockwerk, Balkon und Aufzug. 
Auch die Kehrwoche ist mit einem schwachen positiv linearen Zusammenhang zum Stockwerk dargestellt. 

\useunder{\uline}{\ul}{}
\begin{table}[h]
    \begin{center}
        \begin{tabular}{|l|l|l|l|}
        \hline
            {\ul \textbf{}}              & {\ul \textbf{Stockwerk}} & {\ul \textbf{Kehrwoche}}  & {\ul \textbf{Balkon}} \\ \hline
            {\ul \textbf{Stockwerk}}     & 1                        &                           &                       \\ \hline
            {\ul \textbf{Kehrwoche}}     & 0,31                     & 1                         &                       \\ \hline
            {\ul \textbf{Balkon}}        & 0,53                     & 0,13                      & 1                     \\ \hline
            {\ul \textbf{Aufzug}}        & 0,67                     & 0,23                      & 0,43                  \\ \hline
        \end{tabular}
        \caption{Korrelationsmatrix Stockwerk, Kehrwoche, Balkon, Aufzug}
        \label{tab:korr2}
    \end{center}
\end{table}
 
Wird auch das Bewertungsattribut in die Korrelationsmatrix eingebunden, können bereits erste Vermutungen dazu getroffen werden, 
welche Attribute besonders aussagekräftig, beziehungsweise welche weniger relevant für die Bewertung einer Wohnung sind. 
\autoref{tab:korrBewertung} zeigt die Attribute, die die stärkste Korrelation zur Bewertung besitzen.
\useunder{\uline}{\ul}{}
\begin{table}[h]
    \begin{center}
        \begin{tabular}{|l|l|}
            \hline
            {\ul \textbf{}}              & {\ul \textbf{Bewertung}} \\ \hline
            {\ul \textbf{Schule}}        & 0,51                        \\ \hline
            {\ul \textbf{Zimmerzahl}}    & 0,23                     \\ \hline
            {\ul \textbf{Kindergarten}}  & 0,23                     \\ \hline
            {\ul \textbf{Nebenkosten}}   & 0,19                     \\ \hline
        \end{tabular}
        \caption{Korrelationsmatrix Bewertung und Merkmale}
        \label{tab:korrBewertung}
    \end{center}
\end{table}

Die Schule ist mit einem mittelstarken linearen Zusammenhang sehr aussagekräftig für die Bewertung.
Diese Aussage soll mit den folgenden Programentwürfen und der Implementierung des Entscheidungsbaumes
überprüft werden. 