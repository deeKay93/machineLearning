\section{Neuronales Netz}\label{sec:kaptiel}
\subsection{Einleitung}
Um ein Neuronales Netz zu erstellen, dass die Kundenpräferenzen in einer Wohnungsbörse erlernt,
sind zunächst einige Vorverarbeitungsschritte notwendig.
\begin{itemize}
    \item Analysieren des Datensatzes, um aussagekräftige Eingabewerte für das Neuronale Netz zu
            finden.
    \item Umwandlung des Datensatzes in numerische Werte, damit diese vom Netz konsumierbar sind.
    \item Normierung des Datensatzes um eine Vorgewichtung von Attributen zu vermeiden.
    \item Entscheidung über die Struktur des Netzes, Anzahl der Neuronen und Schichten, und Auswahl 
            eines geeigneten Lernverfahrens.
\end{itemize}
Diese Schritte werden in den folgenden Abschnitten erläutert. 

\subsection{Analysieren des Datensatzes}
Ein Neuronales Netz lernt aus Informationen nicht aus den Daten. Aus diesem Grund ist es wichtig, 
die Informationen die das Netz als Eingabe erhält möglichst genau zu spezifizieren. Eingaben die 
eine hohe Korrelation besitzen können möglicherweise in einem aussagekräftigen Attribute
zusammengefasst werden. 
\autoref{tab:korr1} und \autoref{tab:korr2} zeigen Ausschnitte aus der Korrelationsmatrix aller 
gegebenen Attribute. 
\autoref{tab:korr1} zeigt einen Zusammenhang, der den Erwartungen entspricht. \textit{Eine Wohnung die größer ist, kostet auch mehr}.
Eine Möglichkeit diesen Zusammenhang darzustellen, ist das Attribut \textit{Kosten pro Quadratmeter}. Durch dieses Attribut
könnten die verschiedenen Wohnungen unabhängig der Größe verglichen werden. Der Zusammenhang zwischen \textit{Miete, Kaution und Nebenkosten}
lässt sich in einem Attribut \textit{Gesamtkosten} zusammenfassen.
Auch der Ausschnitt den \autoref{tab:korr2} zeigt, ist nachvollziehbar. Jedoch wird durch die berechneten Korrelationen sichtbar, 
dass Beispielsweise die \textit{Kehrwoche} nicht sehr stark mit dem \textit{Balkon} korreliert. Selbst die Korrelation zwischen
\textit{Stockwerk} und \textit{Aufzug} ist nicht sehr stark. Aus diesem Grund bietet es sich hier nicht an, ein zusammenfassende 
Eingabe zu verwenden.

\useunder{\uline}{\ul}{}
\centering
\begin{table}[h]
    \begin{tabular}{|l|l|l|l|l|}
        \hline
        {\ul \textbf{}}             & {\ul \textbf{Zimmerzahl}} & {\ul \textbf{Miete}} & {\ul \textbf{Nebenkosten}} & {\ul \textbf{Kaution}} \\ \hline
        {\ul \textbf{Zimmerzahl}}   & 1                         &                      &                            &                        \\ \hline
        {\ul \textbf{Miete}}        & 0,96                      & 1                    &                            &                        \\ \hline
        {\ul \textbf{Nebenkosten}}  & 0,94                      & 0,94                 & 1                          &                        \\ \hline
        {\ul \textbf{Kaution}}      & 0,58                      & 0,58                 & 0,58                       & 1                      \\ \hline
        {\ul \textbf{Quadratmeter}} & 0,98                      & 0,96                 & 0,94                       & 0,58                   \\ \hline
    \end{tabular}
    \caption{Ausschnitt 1 Korrelationsmatrix}
    \label{tab:korr1}
\end{table}

\useunder{\uline}{\ul}{}
\begin{table}[h]
    \centering
    \begin{tabular}{|l|l|l|l|}
    \hline
        {\ul \textbf{}}              & {\ul \textbf{Stockwerk}} & {\ul \textbf{Kehrwoche}}  & {\ul \textbf{Balkon}} \\ \hline
        {\ul \textbf{Stockwerk}}     & 1                        &                           &                       \\ \hline
        {\ul \textbf{Kehrwoche}}     & 0,31                     & 1                         &                       \\ \hline
        {\ul \textbf{Balkon}}        & 0,53                     & 0,13                      & 1                     \\ \hline
        {\ul \textbf{Aufzug}}        & 0,67                     & 0,23                      & 0,43                  \\ \hline
    \end{tabular}
    \caption{Ausschnitt 2 Korrelationsmatrix}
    \label{tab:korr2}
\end{table}


\paragraph{Aussagekraft der Attribute}
% Wird auch das Bewertungsattribut in die Korrelationsmatrix eingebunden, lässt sich bereits ein erster 


\subsection{Tensorflow}

\subsection{Konzept zur Implementierung mit Tensorflow}

\subsection{Zusammenfassung}
