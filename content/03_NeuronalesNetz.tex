\section{Neuronales Netz}\label{sec:kaptiel}
\subsection{Einleitung}
\subsection{Vorverarbeitung der Daten}
Die größte Gefahr bei Problemlösungen mit Neuronalen Netzen, ist es das Konzept zu überschätzen. 
Es ist sehr wichtig, die Problemstellung vor dem Entwurf eines Neuronalen Netzes verstanden zu haben. 
Das Netz benötigt Daten, die möglichst aussagekräftige Informationen enthalten. 
Die Kernfrage, die das Netz beantworten soll ist: 
\textit{Welche Wohnungseigenschaften sorgen dafür, dass ein Interesse an der Wohnung entsteht.}
Um dem Netz Daten zu übergeben, mit deren Hilfe diese Frage beantwortet werden kann, ist es zunächst notwendig
den gegebenen Datensatz zu analysieren. Dabei liegt der Fokus auf einer möglichen Korrelation von gegeben Attributen und darauf ob ein Attribut 
zielführend für die Beantwortung dieser Kernfrage ist. 
\paragraph{Korrelation der gegebenen Attribute:}

Durch das Erstellen einer Korrelationsmatrix, lassen sich sehr schnell stark korrelierende Attribute finden.


\paragraph{Aussagekraft der Attribute}


\subsection{Tensorflow}

\subsection{Konzept zur Implementierung mit Tensorflow}

\subsection{Zusammenfassung}
