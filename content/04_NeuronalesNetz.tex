\section{Neuronales Netz}\label{sec:nn}
\subsection{Einleitung}
Um ein Neuronales Netz zu erstellen, dass die Kundenpräferenzen in einer Wohnungsbörse erlernt,
sind zunächst einige Vorverarbeitungsschritte notwendig.
\begin{itemize}
    \item Analysieren des Datensatzes, um aussagekräftige Eingabewerte für das Neuronale Netz zu
            finden.
    \item Umwandlung des Datensatzes in numerische Werte, damit diese vom Netz konsumierbar sind.
    \item Normierung des Datensatzes um eine Vorgewichtung von Attributen zu vermeiden.
    \item Entscheidung über die Struktur des Netzes, Anzahl der Neuronen und Schichten, und Auswahl 
            eines geeigneten Lernverfahrens.
\end{itemize}
Diese Schritte werden in den folgenden Abschnitten erläutert. 

\subsection{Analysieren des Datensatzes}
Ein Neuronales Netz lernt aus Informationen nicht aus den Daten. Aus diesem Grund ist es wichtig, 
die Informationen die das Netz als Eingabe erhält möglichst genau zu spezifizieren. Es ist 
sinnvoller wenige aber sehr aussagekräftige Eingabewerte für das Neuronale Netz zu definieren.
\subsubsection{Korrelation der Eingabewerte} 
Eingaben die eine hohe Korrelation besitzen können möglicherweise in einem aussagekräftigen Attribute
zusammengefasst werden. \\
\autoref{tab:korr1} und \autoref{tab:korr2} zeigen Ausschnitte aus der Korrelationsmatrix aller 
gegebenen Attribute. \\
\autoref{tab:korr1} zeigt einen Zusammenhang, der den Erwartungen entspricht. \textit{Eine Wohnung die größer ist, kostet auch mehr}.
Eine Möglichkeit diesen Zusammenhang darzustellen, ist das Attribut \textit{Kosten pro Quadratmeter}. Durch dieses Attribut
könnten die verschiedenen Wohnungen unabhängig der Größe verglichen werden. Der Zusammenhang zwischen \textit{Miete, Kaution und Nebenkosten}
lässt sich in einem Attribut \textit{Gesamtkosten} zusammenfassen.\\
Auch der Ausschnitt den \autoref{tab:korr2} zeigt, ist nachvollziehbar. Jedoch wird durch die berechneten Korrelationen sichtbar, 
dass Beispielsweise die \textit{Kehrwoche} nicht sehr stark mit dem \textit{Balkon} korreliert. Selbst die Korrelation zwischen
\textit{Stockwerk} und \textit{Aufzug} ist nicht sehr stark. Aus diesem Grund bietet es sich hier nicht an, ein zusammenfassende 
Eingabe zu verwenden.

\useunder{\uline}{\ul}{}
\begin{table}[h]
    \begin{center}
        \begin{tabular}{|l|l|l|l|l|}
            \hline
            {\ul \textbf{}}             & {\ul \textbf{Zimmerzahl}} & {\ul \textbf{Miete}} & {\ul \textbf{Nebenkosten}} & {\ul \textbf{Kaution}} \\ \hline
            {\ul \textbf{Zimmerzahl}}   & 1                         &                      &                            &                        \\ \hline
            {\ul \textbf{Miete}}        & 0,96                      & 1                    &                            &                        \\ \hline
            {\ul \textbf{Nebenkosten}}  & 0,94                      & 0,94                 & 1                          &                        \\ \hline
            {\ul \textbf{Kaution}}      & 0,58                      & 0,58                 & 0,58                       & 1                      \\ \hline
            {\ul \textbf{Quadratmeter}} & 0,98                      & 0,96                 & 0,94                       & 0,58                   \\ \hline
        \end{tabular}
        \caption{Ausschnitt 1 Korrelationsmatrix}
        \label{tab:korr1}
    \end{center}
\end{table}

\useunder{\uline}{\ul}{}
\begin{table}[h]
    \begin{center}
        \begin{tabular}{|l|l|l|l|}
        \hline
            {\ul \textbf{}}              & {\ul \textbf{Stockwerk}} & {\ul \textbf{Kehrwoche}}  & {\ul \textbf{Balkon}} \\ \hline
            {\ul \textbf{Stockwerk}}     & 1                        &                           &                       \\ \hline
            {\ul \textbf{Kehrwoche}}     & 0,31                     & 1                         &                       \\ \hline
            {\ul \textbf{Balkon}}        & 0,53                     & 0,13                      & 1                     \\ \hline
            {\ul \textbf{Aufzug}}        & 0,67                     & 0,23                      & 0,43                  \\ \hline
        \end{tabular}
        \caption{Ausschnitt 2 Korrelationsmatrix}
        \label{tab:korr2}
    \end{center}
\end{table}
 
\subsubsection{Aussagekraft der Attribute}
Wird auch das Bewertungsattribut in die Korrelationsmatrix eingebunden, können bereits erste Vermutungen dazu getroffen werden, 
welche Attribute besonders aussagekräftig, beziehungsweise welche weniger relevant für die Bewertung der Wohnung sind. 
\autoref{tab:korrBewertung} zeigt die Attribute, die die stärkste Korrelation zur Bewertung besitzen.
\useunder{\uline}{\ul}{}
\begin{table}[h]
    \begin{center}
        \begin{tabular}{|l|l|}
            \hline
            {\ul \textbf{}}              & {\ul \textbf{Bewertung}} \\ \hline
            {\ul \textbf{Schule}}        & 0,51                        \\ \hline
            {\ul \textbf{Zimmerzahl}}    & 0,23                     \\ \hline
            {\ul \textbf{Kindergarten}}  & 0,23                     \\ \hline
            {\ul \textbf{Nebenkosten}}   & 0,19                     \\ \hline
        \end{tabular}
        \caption{Ausschnitt 2 Korrelationsmatrix}
        \label{tab:korrBewertung}
    \end{center}
\end{table}
Zu einem ähnlichen, aber detaillierterem Ergebnis, kommt auch der Entscheidungsbaum. 

\subsubsection{Ergebnis der Analyse}
Attribute mit hoher Korrelation untereinander bzw. mit der Bewertung:
\begin{itemize}
    \item \textbf{Kosten pro Quadratmeter:} Miete, Quadratmeter
    \item \textbf{Gesamtkosten:} Nebenkosten, Miete, Kaution
    \item \textbf{Kinderfreundlichkeit:} Schule, Kindergarten
    \item \textbf{Mobilität:} Entfernung
    \item \textbf{Ausstattung:} Möbliert
    \item \textbf{Zimmerzahl}
\end{itemize}

Ausgeschlossene Attribute mit Korrelationswerten unter 0,1 beziehungsweise -0,1 sind:\\
Stockwerk, Heizung, Hausmeister, S-Bahn, Alter, Aufzug, Lage, Küche, Bad, Balkon, Terrasse, Kehrwoche, 
und Garage.

\subsection{Umwandlung der Daten in numerische Werte}
Im nächsten Schritt müssen alle Werte in numerische Werte umgewandelt werden. 
\begin{itemize}
    \item \textit{num-num}: Bei diesen Werten wird ganz einfach der Mittelwert gebildet. 
    \item \textit{über/unter}: Die Randwerte werden als normale Werte angesehen. 
    \item \textit{nah/erreichbar/fern}: Hier bietet es sich an das Attribut als \textit{Entfernung zu ...} umzuschreiben. 
                So kann für \textit{nah} der Minimalwert und für \textit{fern} der Maximalwert verwendet werden.
    \begin{itemize}
        \item nah: 0
        \item erreichbar: 0,5
        \item fern: 1
    \end{itemize}
    \item \textit{Boolean}: Für Werte die entweder erfüllt oder nicht erfüllt sind, wird 0 (nicht erfüllt) und 1 (erfüllt) verwendet.
\end{itemize}

Berechnungen die durchgeführt werden müssen: 
\begin{itemize}
    \item $Miete pro Quadratmeter = Miete/Quadratmeter$
    \item $Gesamtkosten = Miete + Nebenkosten + Kaution$
    \item $Kinderfreundlichkeit = (Schule + Kindergarten)/2$
\end{itemize}

\subsection{Normierung der Daten}
Um eine Normierung der Daten durchzuführen müssen zunächst die Maximalwerte der 
jeweiligen Attribute definiert werden. Die Werte werden anschließend auf Werte 
zwischen 0 und 1 normiert. Dies ist notwendig, da eine Vorgewichtung der Attribute
vermieden werden soll.

\subsection{Bibliothek Tensorflow}
Tensorflow ist eine Python-Bibliothek für verschiedene Anwendungen im Bereich Deep Learning. 
Der Name spiegelt das Grundkonzept dieser Bibliothek wieder. 
\textit{Tensor} bezeichnet den Datensatz der in das Deep Learning System 
eingegeben und verarbeitet wird. \textit{Flow} bezeichnet den Ablauf, 
der Verarbeitung des Datensatzes. \\
Beim Erstellen eines Systems mit Tensorflow werden zwei Phasen 
unterschieden. 
Die Bibliothek bietet neben dem generellen Konzept für Maschine Learning Anwendungen 
viele mathematische Funktionen. 
\cite{tf:2018}

\subsection{Konzept zur Implementierung mit Tensorflow}
\subsubsection{Lernrate und Lernschritt}
Tensorflow bietet verschiedene Verfahren für konstante und variable Lernraten. 
Als Start für eine konstante Lernrate bietet sich der Wert 0,5 an. Jedoch bietet 
die Implementierung einer variablen Lernrate einige Vorteile. Die Lernrate kann während 
eines Lernschritts erhöht oder verringert werden. Dies steigert die Effizienz des Algorithmus. 
\autoref{lst:nnVariableLernRate} zeigt die Implementierung mithilfe von Tensorflow. 
Die Lernrate wird hier alle 100000 Schritt mit der Basis 0,96 erhöht. 

\lstinputlisting[
    float,
    floatplacement=H,
    caption=Implementierung einer variablen Lernrate,
    label=lst:nnVariableLernRate,
    language=Python
]{nnVariableLernRate.py}

Durch die Minimierung des \textit{global step} wird die Lernrate erhöht. 
Die Definition des Lernschrittes ist in \autoref{lst:nnLernSchritt} dargestellt.
Während jedem Lernschritt werden die Gewichte des Netzes optimiert.
Dies wird mit einer Kostenfunktion ermittelt. 
Die \textit{softmax cross entropy with logits} normiert zunächst die Werte auf Werte zwischen 0 und 1 und berechnet im 
Anschluss den Fehler zwischen berechnetem Wert und gefordertem Wert.

\lstinputlisting[
    float,
    floatplacement=H,
    caption=Implementierung des Lernschrittes,
    label=lst:nnLernSchritt,
    language=Python
]{nnLernschritt.py}

\subsubsection{Initiale Gewichte}
Die initialen Gewichte für das Netz werden zufällig gesetzt. Diese werden in der Lernphase 
mit hilfe einer Kostenfunktion angepasst.

\subsubsection{Aufbau des Perceptrons}
Die Anzahl der Schichten wird anhand der Anzahl der Eingabewerte ermittelt. 
Durch die Reduzierung der Eingabeattribute in der Vorverarbeitung sind nun lediglich 6 Neuronen in 
der Eingabe-Schicht notwendig. Eine Handregel besagt die Anzahl der versteckten Schichten sollte mit
$Anzahl Hidden Layer = Eingabeneuronen/2$ berechnet werden. Demnach sind 3 versteckte Schichten notwendig. 
Die Ausgabe des Netzes soll das Attribut \textit{Bewertung} sein. Demnach enthält die 
Ausgabe-Schicht lediglich ein Neuron.\\
\autoref{lst:nnModel} zeigt die Implementierung eines solchen Perceptrons. Die \textit{matmul} Funktion 
berechnet hierbei eine Matrixmultiplikation zwischen vorhergehender Schicht, Gewichten dieser Schicht und der Neigung dieser Schicht.
\lstinputlisting[
    float,
    floatplacement=H,
    caption=Implementierung des Perceptrons,
    label=lst:nnModel,
    language=Python 
]{nnModel.py}

\subsection{Zusammenfassung}
Die Tensorflow-lib eignet sich sehr gut für die Entwicklung von Deep Learning Systemen. \\
Sie bietet sehr viele Funktionalitäten, die allerdings bei kleinen Beispielen, sehr kompliziert 
werden können.\\
Das Problem der Wohnungsbörse ist ein Problem, dass durchaus mit einem neuronalen Netz gelöst werden kann. 


