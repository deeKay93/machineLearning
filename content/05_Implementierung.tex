\section{Implementierung des Entscheidungsbaums}\label{sec:implementierung}
Das Programm implementiert den zuvor beschriebenen Algorithmus in der Programmiersprache Java.
Lediglich um Kommandozeilenparameter angenehmer einzupflegen wird eine Bibliothek verwendet.

\subsection{Aufbau}
Der Aufbau des Programms ist in \autoref{fig:klasse} dargestellt.
Um die Übersichtlichkeit nicht weiter einzuschränken wird auf das Auflisten der Funktionsparameter verzichtet.
In \autoref{appendix:klassendiagramm} ist das gesamte Diagramm nochmals vergrößert abgebildet.

\begin{figure}[ht!]
    \centering
    \resizebox{\textwidth}{!}{
        \input{assets/img/diagramme/klassendiagramm.latex}
    }
    \caption{Aufbau des Programms}
    \label{fig:klasse}
\end{figure}

\subsection{Ausführen des Programms}
Im Anhang an dieses Dokument befindet sich die Datei \enquote{tree.jar}.
Grundlegend kann das Programm, wie in \autoref{lst:jar-aufruf} gezeigt, aufgerufen werden.
Erforderlich ist hierfür zumindest Java in der Version 8.

\begin{lstlisting}[
    caption=Einfacher Aufruf des Programms,
    label=lst:jar-aufruf,
    language=bash
]
java -jar tree.jar
\end{lstlisting}

Das Programm kann durch verschiedene Parameter beeinflusst werden:
Diese sind mit ihren Standartwerte in folgender Übersicht dargestellt.

\begin{description}
    \setlength\itemsep{-0.5em}
    \item[\texttt{-{}-teach, -t}]
        Datei, welche zum Trainieren genutzt wird\\
        Standard: \texttt{Wohnungskartei\_Muster\_Master\_5\_K\_teach.csv}
    \item[\texttt{-{}-check, -c}]
        Datei, welche mit dem Baum geprüft wird\\
        Standard: <leer>
    \item[\texttt{-{}-print, -p}]
        Wenn gesetzt, wird der Baum wie in \autoref{appendix:baum-training} dargestellt\\
        Standard: false
    \item[\texttt{-{}-default, -d}]
        Initiale Default Wahrscheinlichkeit\\
        Standard: 0,5
    \item[\texttt{-{}-min-ratio, -r}]
        Mindestverhältnis zwischen Werten eines Attributs und Beispielmenge zur weiteren Unterteilung.\\
        Standard: 1
    \item[\texttt{-{}-use-saved-tree, -u}]
        Wenn gesetzt, wird der Baum aus der angegebenen Datei geladen und nicht neu erzeugt.\\
        Standard: <leer>
    \item[\texttt{-{}-save-tree, -s}]
        Speichert den neu trainierten Baum in der angegebenen Datei.\\
        Standard: <leer>
    \item[\texttt{-{}-help}]
        Gibt die Hilfe in der Kommandozeile aus
\end{description}



\subsection{Besondere Aspekte}
\paragraph{Attribute}
Alle verfügbaren Attribute werden als \lstinline[language=Java]{enum} modelliert.
Jedes \lstinline[language=Java]{enum} gibt über die Funktion
\lstinline[language=Java]{getWerte()} die möglichen Ausprägungen als Liste von Zeichenketten zurück.

Auf diese Weise kann das entsprechende Attribut als Parameter in Funktionsaufrufen übergeben werden und
innerhalb dieser Funktion kann -- falls notwendig -- auf die möglichen Attributwerte zugegriffen werden.

\paragraph{Einlesen der Daten}
Weitestgehend werden die Werte der Datensätze direkt eingelesen und in ein \emph{Datensatz}-Objekt überführt.
Dieses Objekt enthält je ein Attribut für jedes der Merkmale und die Bewertung.
Für das Merkmal \enquote{Heizung} und für die Bewertung sind jedoch Schritte zur Vorverarbeitung notwendig:

Aus dem Zeichenketten \enquote{ja} und \enquote{nein} der Bewertung wird der entsprechende boolesche Wert ermittelt.
In der Attributausprägung \enquote{Lueftungsheizung} enthält der Datensatz anstatt des \textit{ue} ein undefiniertes Zeichen.
Dieses Problem wird dadurch gelöst, dass der Wert wie in \autoref{lst:heizung} angepasst wird.
Zwar wäre auch eine einmalige Änderung der entsprechende Werte in der Beispieldatei möglich,
die Programmzeile ermöglicht es aber auch, dass andere Dateien mit gleicher Formatierung bedenkenlos genutzt werden können.

\begin{lstlisting}[
    firstnumber=54,
    stepnumber=54,
    caption={Anpassen von \enquote{Heizung}},
    label=lst:heizung,
    language=Java
]
heizung = heizung.endsWith("ftungsheizung") ?
                    "Lueftungsheizung" :
                    heizung;
\end{lstlisting}

\paragraph{Datensatz}
Ein Datensatz-Objekt enthält für jeden Merkmalstyp eine Zeichenkette mit entsprechenden Namen.
Um dynamisch den Wert eines bestimmten Attributs zu erhalten kann dieses über die Funktion \lstinline[language=Java]{get()}
in \autoref{lst:datensatz-get} abgefragt werden.

\lstinputlisting[
    caption={\texttt{get}-Funktion des Datensatzes (gekürzt)},
    label=lst:datensatz-get,
    language=Java
]{datensatz_get.java}

Da die Bewertung ein boolescher Wert ist, die Funktion aber eine Zeichenkette zurückgegeben muss,
wird dieser Wert entsprechend in eine Konstante umgewandelt.
Für alle anderen Attribute wird der entsprechende Wert des Datensatzes zurückgegeben.

\paragraph{Spezialisierung}
Das vorliegende Programm ist vor alle durch die Attribute und den Datensatz stark auf den gegebenen Anwendungsfall spezialisiert.
Theoretisch wäre es möglich ein Programm zu so konzipieren, dass es die Attribute und deren Wertemenge selbstständig aus den vorliegenden Daten extrahiert.
Dies würde jedoch die Komplexität und damit die Fehleranfälligkeit des Programms deutlich erhöhen.

\paragraph{Rekursionen}
Zum Erstellen des Baumes, aber auch zu Evaluation eines Datensatzes und zum Darstellen des Baumes wird Rekursion genutzt.
Beim Erstellen wird ein Attribut ausgewählt und für jede Ausprägung ein Unterbaum nach gleichem Muster gebildet bis eine Ende-Bedingung erreicht ist.
Details hierzu sind in \autoref{alg:baum-erzeugen} dargestellt.

Die Ausgabe des Baums geschieht,
indem erst die Daten des aktuellen Knotens und
dann nach und nach die Daten der Unterbäume zu einer Zeichenkette hinzugefügt werden.

Um den Wert eines neuen Datensatzes zu ermitteln wird ausgehend von der Wurzel in jedem Knoten der zugehörige Attributwert aus dem Datensatz ausgelesen.
Dann wird die Bestimmungsfunktion des zu diesem Wert zugehörige Unterbaums aufgerufen.
Dies geschieht rekursiv bis ein Blattknoten erreicht und damit der Wert ermittelt ist.

\subsection{Test und Ergebnisbewertung}
Um den Algorithmus zu Testen und dessen Ergebnisse zu bewerten wird er in verschiedenen Konfigurationen ausgeführt (siehe \autoref{tab:run-config}).


\useunder{\uline}{\ul}{}
\begin{table}[h]
    \begin{center}
        \begin{tabular}{|l|l|l|l|}
        \hline
                               & {\ul \textbf{Trainingsdaten}} & {\ul \textbf{Testdaten}}      & {\ul \textbf{Verhältnis}} \\
            \hline
            T0                  & \dots 5\_K\_teach.csv & \dots 4\_K.csv        & $0$ \\
            \hline
            T1                  & \dots 5\_K\_teach.csv & \dots 4\_K.csv        & $1$ \\
            \hline
            T2                  & \dots 5\_K\_teach.csv & \dots 4\_K.csv        & $2$ \\
            \hline
            \hline
            C0                  & \dots 4\_K.csv        & \dots 5\_K\_teach.csv & $0$ \\
            \hline
            C1                  & \dots 4\_K.csv        & \dots 5\_K\_teach.csv & $1$ \\
            \hline
            C2                  & \dots 4\_K.csv        & \dots 5\_K\_teach.csv & $2$ \\
            \hline
        \end{tabular}
        \caption{Konfigurationen zur Ausführung}
        \label{tab:run-config}
    \end{center}
\end{table}

\paragraph{Genauigkeit der Klassifikation}
Die Genauigkeit der Klassifikation der jeweiligen Bäume wird bestimmt,
indem sowohl für die Trainings- als auch für die Testdaten eine Klassifikation ermittelt wird.
Die Ist-Klassifikation wird anschließend mit der Soll-Klassifikation verglichen.
Als Ergebnis wird der Anteil der korrekt klassifizierten Datensätze ausgegeben.
Da das Ergebnis der Klassifikation eine Zahl zwischen $0$ und $1$ ist,
gilt jeder Wert $\geq 0.5$ als positive und jeder Wert $< 0.5$ als negative Klassifikation.
Es ergeben sich die in \autoref{tab:run-results} dargestellten Ergebnisse:

\useunder{\uline}{\ul}{}
\begin{table}[h]
    \begin{center}
        \begin{tabular}{|l|l|l|l|}
        \hline
                                & {\ul \textbf{Ergebnis Trainingsdaten}} & {\ul \textbf{Ergebnis Testdaten}} \\
            \hline
            T0                  & $100\%$                                & $\approx 98,901\%$                  \\
            \hline
            T1                  & $99,75\%$                              & $\approx 98,901\%$                  \\
            \hline
            T2                  & $99\%$                                 & $\approx 98,601\%$                  \\
            \hline
            \hline
            C0                  & $100\%$                                & $98,5\%$                  \\
            \hline
            C1                  & $100\%$                                & $98,5\%$                  \\
            \hline
            C2                  & $\approx 99,4\%$                       & $97,5\%$                  \\
            \hline
        \end{tabular}
        \caption{Anteil korrekt klassifizierter Datensätze}
        \label{tab:run-results}
    \end{center}
\end{table}

Die Durchläufe mit Mindestverhältnis $2$ (T2, C2) scheiden im Vergleich sowohl bei den Test- als auch bei den Trainingsdaten am schlechtesten ab.
Bei einem Mindestverhältnis von $0$ (T0, C0) ist das Abbruchkriterium quasi nicht vorhanden.
In diesem Fall werden die Trainingsdaten exakt nachgebildet,
was bedeutet, dass alle Trainingsdaten korrekt klassifiziert werden.

Im Fall der Konfiguration 2 mit einem Mindestverhältnis von $1$ (T1, C1) werden, verglichen mit Konfiguration 1, in den Testdaten $0.25\%$ weniger Datensätze korrekt klassifiziert.
Der Anteil der korrekt klassifizierten Testdaten bleibt jedoch gleich.
Daraus lässt sich schließen,
dass der Baum allgemeiner ist, was einer Überanpassung entgegenwirkt.

\paragraph{Fehleranalyse}
In \autoref{tab:run-results} wird gezeigt, dass die Klassifikation insgesamt ein zufriedenstellendes Ergebnis liefert.
Je nach Anwendungsfall ist jedoch zu unterscheiden,
ob ein falsch klassifizierter Datensatz ein \emph{False Positive} oder ein \emph{False Negative} ist.

In \autoref{tab:fehleranalyse} sind die tatsächlichen Klassifikationen von T1 und C1 dargestellt.
Für T1 gibt es lediglich \emph{False Postives}, jedoch sowohl für die Trainings,
als auch die Testdaten.
Im Fall von C1 werden alle Trainingsdaten korrekt klassifiziert,
allerdings entstehen in den Testdaten sowohl \emph{False Positives}, als auch ein \emph{False Negative}.

Welcher der beiden Fehler schwerer wiegt, hängt vom Anwendungsfall ab.
Geht man in dem gegebenen Beispiel von einem Makler aus,
so sind vermutlich \emph{False Negatives} gravierender.
Bietet der Makler diese Wohnung dem Kunden nicht an,
so kann der Kunde sich logischerweise nicht für sie entscheiden.

Da die Klassifikation keine \enquote{Ja-Nein}-Entscheidung trifft,
sondern eine Zahl zwischen $0$ und $1$ zurückgibt,
kann die Art des Fehlers beeinflusst werden.
Nimmt man beispielsweise an,
dass ein Datensatz bereits ab $0,3$ als positiv klassifiziert wird,
so steigt die Anzahl der \emph{False Positives}.

\useunder{\uline}{\ul}{}
\begin{table}[h]
	\begin{center}
		\begin{tabular}{|c|c|c|}
			\hline
			                        & {\ul \textbf{T1}} & {\ul \textbf{C1}}   \\
			\hline
			{\ul \textbf{Trainingsdaten}}       &
			{        \begin{tabular}{|l|l|l|}
			\hline
			{\ul \textbf{Ist/Soll}} & {\ul \textbf{Ja}} & {\ul \textbf{Nein}} \\
			\hline
			{\ul \textbf{Ja}}       & $68$              & $1$                 \\
			\hline
			{\ul \textbf{Nein}}     & $0$               & $331$               \\
			\hline
		\end{tabular}}
		&            {
        \begin{tabular}{|l|l|l|}
			\hline
			{\ul \textbf{Ist/Soll}} & {\ul \textbf{Ja}} & {\ul \textbf{Nein}} \\
			\hline
			{\ul \textbf{Ja}}       & $65$              & $0$                 \\
			\hline
			{\ul \textbf{Nein}}     & $0$               & $936$               \\
			\hline
		\end{tabular}}                 \\
		\hline
		{\ul \textbf{Testdaten}}     &            {
        \begin{tabular}{|l|l|l|}
			\hline
			{\ul \textbf{Ist/Soll}} & {\ul \textbf{Ja}} & {\ul \textbf{Nein}} \\
			\hline
			{\ul \textbf{Ja}}       & $65$              & $11$                 \\
			\hline
			{\ul \textbf{Nein}}     & $0$               & $925$               \\
			\hline
		\end{tabular}}               &            {
         \begin{tabular}{|l|l|l|}
			\hline
			{\ul \textbf{Ist/Soll}} & {\ul \textbf{Ja}} & {\ul \textbf{Nein}} \\
			\hline
			{\ul \textbf{Ja}}       & $67$              & $5$                 \\
			\hline
			{\ul \textbf{Nein}}     & $1$               & $327$               \\
			\hline
		\end{tabular}}               \\
		\hline
		\end{tabular}
		\caption{Fehleranalyse}
		\label{tab:fehleranalyse}
	\end{center}
\end{table}

\paragraph{Auswirkungen des Mindestverhältnis}
Die Entscheidungsbäume von T0 und T1 unterscheiden sich im Unterbaum
\emph{Schule (nah)} $\rightarrow$ \emph{Moebliert (nein)} $\rightarrow$ \emph{Zimmerzahl (4)} $\rightarrow$ \emph{Kindergarten (nah)} $\rightarrow$ \emph{Kehrwoche (ja)}.
\autoref{tab:diff} zeigt diesen Unterschied.

\useunder{\uline}{\ul}{}
\begin{table}[h]
    \begin{center}
        \begin{tabular}{|p{6cm}|p{6cm}|}
        \hline
            {\ul \textbf{T0}}   & {\ul \textbf{T1}} \\
            \hline
            {\lstinputlisting[
                language={},
                numbers=none,
                backgroundcolor=\color{white},
                frame=none
            ]{unterbaum_t0.txt}}
            &
            {\lstinputlisting[
                language={},
                numbers=none,
                backgroundcolor=\color{white},
                frame=none
            ]{unterbaum_t1.txt}}
            \\
            \hline
        \end{tabular}
        \caption{Unterschiede bei Mindestverhältnis von $0$ und $1$}
        \label{tab:diff}
    \end{center}
\end{table}

Betrachtet man die eigentlichen Trainingsdaten, so finden sich dort lediglich zwei Datensätze.
Diese unterscheiden sich in den Attributen \emph{Stockwerk, Heizung, S-Bahn, Miete, Nebenkosten, Alter, Lage, Entfernung, Kaution, Bad, Balkon} und \emph{Quadratmeter}
Die Auswahl von \emph{Stockwerk} in T0, basiert also nicht auf dem besten Merkmal,
 sondern auf der Reihenfolge der Merkmale.
Mit lediglich zwei Datensätzen, welche sich in so vielen Attributen voneinander unterscheiden,
kann nicht mit Sicherheit bestimmt werden welches Attribut ausschlaggebend für die Entscheidung des Kunden ist.
Um eine Überanpassung zu verhindern ist es daher sinnvoll diesen Unterbaum nicht weiter zu verzweigen.

\paragraph{Interpretation des Baumes}
Für T1 ergibt sich der in \autoref{appendix:baum-training} dargestellte Baum.
Er zeigt folgende Eigenschaften der Trainingsdaten:
\begin{itemize}
    \item Für die Kunden ist es unumgänglich, dass die Wohnung in Schulnähe und nicht möbliert ist.
    \item Es kommen nur Wohnungen zwischen 3-4 und 5 Zimmern in Frage.
    \item Bei 3-4 bis 4-5 Zimmern darf die Wohnung nicht zu weit von einem Kindergarten entfernt sein.
    \item Ist eine 4-5 Zimmerwohnung in erreichbarer Nähe eines Kindergartens, so sollte sie nicht nur ein Waschbecken enthalten.
    \item Bei 5 Zimmern darf die Wohnung nicht über 120 Quadratmeter groß sein.
\end{itemize}

Aus diesen Eigenschaften lässt sich Interpretieren,
dass der Kunde eine Familie mit Schul- und wahrscheinlich Kindergartenkind(ern) ist.
Hauptindizien hierfür sind, die Nähe zu Schule und Kindergarten, sowie die Zimmerzahl.


Betrachtet man den Baum von C1 in \autoref{appendix:baum-test},
so finden sich diese Merkmale weitestgehend wieder.
Auch hier ist die Nähe zur Schule das wichtigste Merkmal.
Allerdings folgt als nächstes Merkmal die Zimmerzahl (wieder mit 3-4 bis 5 Zimmern).
Erst anschließend folgen - je nach Zimmerzahl in unterschiedlichen Reihenfolgen - die Kindergartennähe,
das Bad, und die Möblierung.
Interessanterweise kommen in zwei Fällen die Attribute Kaution beziehungsweise die Art der Heizung hinzu.
