\section{Einleitung}\label{sec:einleitung}
In diesem Programmentwurf wird das Problem \emph{Erlernen von Kundenpräferenzen in einer Wohnungsbörse}
bearbeitet. Gegeben sind zwei Mengen von Datensätzen:
\begin{itemize}
    \item Trainingsdaten zum Trainieren das Algorithmus
    \item Testdaten zur Überprüfung des trainierten Algorithmus
\end{itemize}

Die Datensätze beschreiben 22 Merkmale von Wohnungen,
die in einer Wohnungsbörse angeboten werden.

Zunächst werden die gegebenen Daten analysiert.
Erkenntnisse aus der Analyse werden genutzt um den Algorithmus für einen Entscheidungsbaum zu modellieren.
Als ein alternatives Konzept wird der Entwurf eines neuronalen Netzes vorgestellt.
Hierfür werden unteranderem die Daten vorverarbeitet und die Struktur und Arbeitsweise des neuronalen Netzes definiert.
Das Netz wird allerdings nicht praktisch getestet.

Um das Problem der Wohnungsbörse zu lösen wird abschließend der Entscheidungsbaum implementiert.
Er wird mit den Trainingsdaten aufgebaut und mit den Testdaten validiert.
Die Ergebnisse und Erkenntnisse werden schließlich zusammengefasst und bewertet.