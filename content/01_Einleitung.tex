\section{Einleitung}\label{sec:einleitung}
In diesem Programmentwurf wird das Problem \emph{Erlernen von Kundenpräferenzen in einer Wohnungsbörse}
bearbeitet. Gegeben sind zwei Datensätze: 
\begin{itemize}
    \item Datensatz zum Trainieren mit erwartetem Ergebnis
    \item Datensatz zum Testen ohne erwartetes Ergebnis
\end{itemize}

Die Datensätze beschreiben 22 Merkmale von Wohnungen die in einer Wohnungsbörse angeboten werden. \\
Zunächst werden die gegebenen Daten analysiert. Anschließend werden die aus der Analyse gewonnenen Informationen 
in ein Konzept für einen Entscheidungsbaum einfließen. 
Als ein alternatives Konzept wird der Entwurf eines neuronalen Netzes vorgestellt. Hierfür werden unteranderem
die Daten vorverarbeitet und die Struktur und Arbeitsweise des neuronalen Netzes vorgestellt. Das Netz allerdings nicht praktisch getestet. 
Um das Problem der Wohnungsbörse zu lösen wird abschließend der Entscheidungsbaum implementiert und optimiert. 
Der Entscheidungsbaum wird mit den Trainingsdaten aufgebaut und mit den Testdaten validiert.
Die Ergebnisse und Erkenntnisse werden abschließend zusammengefasst und bewertet.