\section{Fazit}\label{sec:fazit}
\subsection{Konzepte}


\subsection{Ergebnis und Implementierung}
\todo[inline]{Bin mir nicht sicher, ob die subsection-Einteilung so Sinn macht...}

Durch die Klassifikation eines Datensatzes werden Ergebnisse der Klassifikation untereinander vergleichbar.
So würde der Kunde (anhand der Trainingsdaten) eine  \emph{schulnahe, unmöbelierte, 4-5-Zimmer-Wohnung in erreichbarer Kindergartennähe mit Badewanne} wahrscheinlicher als positiv bewerten,
als eine \emph{schulnahe, unmöbelierte, 5-Zimmer-Wohung mit 71-80 Quadratmetern}.

Die \emph{Verhältnisoptimierung} wirkt einer Überanpassung des Baumes entgegen.
Zwar hat ein Mindestverhältnis von $1$ in den gegebenen Daten keinen Einfluss auf die Klassifikation der Testdaten und sogar einen negativen Einfluss auf die Trainingsdaten.
Allerdings zeigt sich beim direkten Verlgeich der Bäume,
dass durch das Verhältnis der Baum weniger tief und damit allgemeiner ist.

Allerdings kann diese Optimierung auch dazu führen,
dass alle Blattknoten eines Baumes das gleiche binäre Ergebnis liefern.
Ist lediglich eine binäre Klassifikation gewünscht,
so sollte der Baum nachträglich \enquote{gesäubert} werden,
um die spätere Klassifikation zu erleichtern.

Insgesamt zeigt die Implementierung und deren Ausführung mit den gegebenen Daten,
dass in allen Fällen - mit Ausnahme von C2 - die Testdaten mit über $98\%$ erfolgreich klassifiziert wurden.
Außerdem ist der \enquote{gedruckte} Baum für einen Menschen greifbar und einfach zu interpretieren.