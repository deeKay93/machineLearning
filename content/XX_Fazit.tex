\section{Fazit}\label{sec:fazit}
\paragraph{Neuronales Netz}
In diesem Programmentwurf wurde das Konzept für ein neuronales Netz vorgestellt,
welches das Wohnungsbörsenproblem löst.
Bei einem Konzept für ein neuronales Netz ist die Vorverarbeitung ein sehr wichtiger und nicht zu vernachlässigender Schritt.
Der nächste Schritt ist es nun, dieses Konzept praktisch zu testen.
Beispielsweise ist eine Umsetzung mit der Python-Bibliothek \emph{Tensorflow} denkbar.
Tensorflow bietet verschiedene Methoden, mit deren Hilfe Deep Learning Systeme erstellt werden können.
Einige Parameter wie zum Beispiel die Lernrate können im Rahmen einer praktischen Umsetzung weiter optimiert werden. \\
Generell lässt sich feststellen, dass ein neuronales Netz die gegebene Problemstellung lösen kann.

\paragraph{Ergebnis und Implementierung}
Durch die Klassifikation eines Datensatzes als numerischen Wert werden die Ergebnisse der Klassifikation untereinander vergleichbar.
So würde der Kunde (anhand der Trainingsdaten) eine  \emph{schulnahe, unmöblierte, 4-5-Zimmer-Wohnung in erreichbarer Kindergartennähe mit Badewanne} wahrscheinlicher als positiv bewerten,
als eine \emph{schulnahe, unmöblierte, 5-Zimmer-Wohnung mit 71-80 Quadratmetern}.

Die \enquote{Verhältnisoptimierung} wirkt einer Überanpassung des Baumes entgegen.
Zwar hat ein Mindestverhältnis von $1$ in den gegebenen Daten keinen Einfluss auf die Klassifikation der Testdaten und sogar einen negativen Einfluss auf die Trainingsdaten.
Allerdings zeigt sich beim direkten Vergleich der Bäume,
dass durch das Verhältnis der Baum weniger tief und damit allgemeiner ist.

Diese Optimierung kann allerdings auch dazu führen,
dass alle Blattknoten eines Baumes das gleiche binäre Ergebnis liefern.
Ist lediglich eine binäre Klassifikation gewünscht,
so sollte der Baum nachträglich \enquote{gesäubert} werden,
um die spätere Klassifikation zu erleichtern.

Insgesamt zeigt die Implementierung und deren Ausführung mit den gegebenen Daten,
dass in allen Fällen - mit Ausnahme von C2 - die Testdaten mit über $98\%$ erfolgreich klassifiziert wurden.
Außerdem ist der \enquote{gedruckte} Baum für einen Menschen greifbar und einfach zu interpretieren.